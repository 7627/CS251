
\documentclass[a4paper, 10pt,twocolumn]{article}
\usepackage{amsmath}
\usepackage{lipsum}
\usepackage[procnumbered,ruled,linesnumbered]{algorithm2e}
\usepackage{algorithmic}
%\usepackage[noend]{algpseudocode}
\usepackage{xpatch}
\usepackage{listings}
\setlength{\parindent}{1em}
\title{Comparison Based Sorting Algorithms}
\author{kamlesh }
\date{}

\makeatletter
\def\BState{\State\hskip-\ALG@thistlm}
\makeatother

\begin{document}

\maketitle

\section*{Abstract}

This document presents a brief discussion on sort-
ing algorithms. Algorithms for Quicksort is pro-
vided in this document and its working is explained.
Further, a proof of lower bounds on sorting is pre-
sented in this document. Most of the content pre-
sented here is created by referring and reproducing
contents from one of the widely followed book on
Algorithms by Cormen et al\cite{first}.\textbf{We do not claim
originality of this work.} This document is pre-
pared as part of an assignment for the Software Lab
Course (CS251) to learn \LaTeX{}
\noindent\fbox{%
    \parbox{.45\textwidth}{%
       Declaration: I have prepared this document us-
ing \LaTeX{}without any unfair means. Further,
while the document is prepared by me, I do not
claim the ownership of the ideas presented in
this document.
    }%
}
\section{Introduction}
Sorting is one of the most fundamental operations
in computer science useful in numerous applica-
tions. Given a sequence of numbers as input, the
output should provide a non-decreasing sequence
of numbers as output. More formally, we define a
sorting problem as follows \cite{first},\\
\textbf{Input:} A sequence of n numbers $\langle a_1 ,a_2,...,a_n\rangle$.
\textbf{Output:}\hspace{.6cm}A\hspace{.3cm} reordered \hspace{.1cm}sequence\hspace{.5cm} (of size n)\\
$\langle a_1^{'} ,a_2^{'},...,a_n^{'}\rangle$of the input sequence such that$a_1^{'}\leq a_2^{'}\leq ...\leq a_n^{'}$.\\
Consider the following example. Given an input
sequence $\langle8, 34, 7, 9, 15, 91, 15\rangle$, a sorting algorithm
should return $\langle7, 8, 9, 15, 15, 34, 91\rangle$ as output.

A fundamental problem like sorting has attracted
many researchers who contributed with innovative
algorithms to solve the problem of sorting effi-
ciently\cite{three}. Efficiency of an algorithm depends on
primarily on two aspects,

\begin{itemize}
    \item \textbf{Time complexity} is a formalism that cap-tures running time of an algorithm in terms of the input size. Normally, asymptotic behavior on the input size is used to analyze the time complexity of algorithms.
    \item \textbf{Space complexity}is a formalism that cap-
tures amount of memory used by an algorithm
in terms of input size. Like time complexity
analysis, asymptotic analysis is used for space
complexity.
\end{itemize}
In the branch of algorithms and complexity in com-
puter science, space complexity takes a back seat
compared to time complexity. Recently, another
parameter of computing i.e., energy consumption
has become popular. Roy et al.\cite{four} proposed an en-
ergy complexity model for algorithms. In this doc-
ument, we will deal with time complexity of sorting
algorithms.

One class of algorithms which are based on ele-
ment comparison are commonly known as compar-
ison based sorting algorithms. In this document we
will provide a brief overview of Quicksort, a com-
monly
used comparison based sorting algorithm\cite{two}.
Quicksort\hspace{.1mm} is a sorting algorithm based on divide-and-conquer
paradigm of algorithm design. Fur-ther,
we
will
derive the lower bound of any com-parison based sorting algorithm to be $\Omega( n \log_2 n)$for
an input size of n.

\section{Quicksort}
Quicksort is designed as a three-step divide-and-
conquer process for sorting an input sequence in
an array\cite{first}. For any given subarray, A[i..j], the
process is as follows,\\
\textbf{Divide:}The array A[i..j] is partitioned into two
subarrays A[i..k] and A[k + 1..j] such that all ele-
ments in A[i..k] is less than or equal to all elements
in A[k + 1..j]. A partitioning procedure is called to
determine k such that at the end of partitioning,
the element at the k th position (i.e., A[k]) does not
change its position in the final output array.

%\noindent\rule{7.8cm}{1pt}
%\textbf{Algorithm 1 }Partition procedure of Quicksort algorithm.


%\noindent\rule{7.8cm}{.5pt}

\begin{algorithm}
\caption{Partition procedure of Quicksort algorithm}\label{euclid}


 \textbf{procedure} Partition(A, i, j)\\
 A is an array of N integers, $A[1..N ]$\\
 i and j are the start and end of subarray\\
 $x \leftarrow A[i]$ \\
 $y \leftarrow i-1$\\
 $z \leftarrow j+1$\\
 \While{$(true)$}{
     $z \leftarrow z-1$\\
     \While{$A[z] > x$}{
         $z \leftarrow z-1$\\
      }
      $y \leftarrow y+1$\\
      \While{$A[y]<x$}{
         $y \leftarrow y+1$\\
      }
     \uIf{$y<z$}{
\         Swap $A[y] \leftrightarrow A[z]$
     }
      \Else{
         return $z$
      }
  }


\end{algorithm}



 \textbf{Conquer:}Conquer:Recursively invoke Quicksort on the two subarrays. This procedure conquers the complexity by applying the same operations in two subarrays.\\
\textbf{Merge:} Quicksort does not require merge or combine operation as the entire array A[i..j] is sorted in place.

In the heart of Quicksort, there is a partition procedure as shown in Algorithm 1. A pivot element x is selected. The first inner while loop (line $#$6) continues examining elements until it finds an element that is smaller than or equal to the pivot element.Similarly, the second inner while loop (line $#9$) element that is greater than or equal to the pivot element. If indices y and z have not exchanged their side around the pivot, the elements at A[y] and A[z] are exchanged. Otherwise, the procedure returns the index z, such that all elements to the left of z are smaller than or equal to A[z] and all elements to the right of z are greater than or equal to A[z].

 \begin{algorithm}
 \caption{Quicksort recursion}\label{euclid}
 
 \textbf{procedure} QUICKSORT(A, i, j)\\
 Quicksort procedure called with A, 1, N\\
 \uIf{$i<j$}{
 $k\leftarrow PARTITION(A,i,j) $
    QUICKSORT(A, i, k)\\
    QUICKSORT(A, k+1, j)\\
 }
 

 \end{algorithm}


The main recursive procedure for Quicksort is presented in Algorithm 2. Initial invocation is per-
formed by call QUICKSORT(A, 1, N ) where N is
the length of array N.
\subsection{Time complexity
Quicksort}
The worst case of Quicksort occurs when an ar-
ray of length N , gets partitioned into two subarrays
of size N-1 and 1 in every recursive invocation of
QUICKSORT procedure in Algorithm 2. The par-
titioning procedure presented in Algorithm 1, takes
Θ(n) time, the recurrence relation for running time
is,\\

$$ T(n)=T(n-1)+\Theta(n)$$\\
As it is evident that$ T(n)=\Theta (n) $, the recurrence is solved as follows,

$$T(n)=T(n-1)+\theta(n)$$\\
$$= \sum_{k=1}^{n}\Theta(k)$$
$$=\Theta\Bigg(\sum_{k=1}^{n} k\Bigg)$$
$$=\Theta(n^2)$$
Therefore, if the partitioning is always maximally unbalanced,
the running time is Θ(n 2 ). Intutively, if an input sequence is almost sorted, Quicksort will perform poorly. In the best case, partitioning divides the array into two equal parts. Thus, the recurrence for the best case is given by,

$$T(n)=2T(\frac{n}{2})+\Theta(n)$$
which evaluates to $\Theta(n \log_2 n)$. Using a comparatively involved analysis, the average running time of Quicksort can be determined to be O(n lg n).
\section{Lower bounds on comparison sorts}
An interesting question about sorting algorithms
based on comparisons is the following: What is
the lower bound of this class of sorting algo-
rithms? This question is important for algorithm
researchers to further improve the sorting algorithms.

A decision tree based analysis leads to the following theorem\cite{one}.
\textbf{Theorem 1.}Any decision tree that sorts n ele- [2] Hoare, C. A. R. Algorithm 64: Quicksort.
ments has height $\omega(n log 2 (n).$\\
\textbf{Proof.} Consider a decision tree of height h that sorts n elements. Since there are n! permutations of n elements, each permutation representing a dis-tinct sorted order, the tree must have at least n$!$
leaves. Since a binary tree of height h has no more than $ 2^h$ leaves. So,\\
$n!\leq 2^h$\\
Applying logarithmic$(\log_2)$the inequality be-
comes,\\
$h \geq log (n!).$\\
Applying Stirling’s approximations,\\
$n! > \Big(\frac{n}{e} \Big)^n$\\
where e is natural base of logarithms. Further,\\
$$h \geq \log  \Big(\frac{n}{e} \Big)^n$$
$$=n \log n - n \log e$$
$$=\Omega (n \log n)$$

\section{Conclusion}
In this document, we have provided a discussion
on sorting algorithms. We included algorithms for
Quicksort and explained its working. Further, a
proof of lower bounds on sorting is presented in this
document. Most of the content presented here is
created by referring and reproducing contents from
one of the widely followed book on Algorithms by
Cormen et al.\cite{first}. We do not claim originality of
this work. This document is prepared as part of an
assignment for the Software Lab Course (CS251) to
learn \LaTeX{}



\begin{thebibliography}{9}
\bibitem{first}
CORMEN, T. H., LEISERSOH, C. E., RIVEST,R. L.,AND STEIN, C. Introduction to Algorithms, Third Edition,
\textit{al.}.
3rd ed. The MIT Press,2009.

\bibitem{two}
HOARE, C. A. R. Algorithm 64: Quicksort
\textit{Communications of ACM 4, 7 (1961), 321–.}
\textit{}


\bibitem{three}
MARTIN, W. A. Sorting.ACM
\textit{Computing Survey 3, 4 (1971), 147–174}
\textit{}


\bibitem{four}
ROY, S., RUDRA, A.,AND VERMA, A.
\texttt{Proceedings of the 4th Conference on Innovations in Theoretical Computer Science (2013),}
ITCS ’13, pp. 283–304.
\end{thebibliography}


\end{document}
